Regressione Lineare. Il componente permette di effettuare la regressione lineare.

Prende 6 segnali dati in ingresso e, attraverso l'utilizzo di moltiplicatori e sottrattori, oltre all'opportuno troncamento dei valori intermedi calcolati, restituisce i parametri di uscita m ed q, rispettivamente coefficiente angolare e intercetta della retta di regressione. La rappresentazione dei segnali è in signed fixed point. Lo schema a blocchi dell'interfaccia del componente è riportata di seguito.  \subsubsection*{Ingressi}


\begin{DoxyItemize}
\item clock\+: segnale di clock, fornisce il segnale di temporizzazione ai componenti interni
\item load\+: segnale di load, agisce solo sui registri di bufferizzazione dei segnali dati in ingresso; si veda la documentazione dell'architettura implementativa
\item reset\+\_\+n\+: segnale di reset asincrono (active-\/low) per i registri interni
\item prim\+: costante in input, 6 bit di parte intera e 0 decimale (m.\+n = 5.\+0)
\item A\+: 6 bit di parte intera e 18 decimale (m.\+n = 5.\+18)
\item B\+: msb di peso -\/1 (m.\+n = -\/1.\+24)
\item C\+: msb di peso -\/7 (m.\+n = -\/7.\+30)
\item Sum1\+: 9 bit di parte intera e 15 decimale (m.\+n = 8.\+15)
\item Sum2\+: 3 bit di parte intera e 21 decimale (m.\+n = 2.\+21)
\end{DoxyItemize}

\subsubsection*{Uscite}


\begin{DoxyItemize}
\item m\+: coefficiente angolare della retta di regressione, 11 bit di parte intera e 13 decimale (m.\+n = 10.\+13)
\item q\+: intercetta della retta di regressione, 3 bit di parte intera e 21 decimale (m.\+n = 2.\+21)
\end{DoxyItemize}

\subsubsection*{Rappresentazione dei segnali}

La rappresentazione dei segnali A, B, C e prim è calzante con i valori costanti degli stessi, forniti per effettuare il test del componente. \begin{TabularC}{3}
\hline
\rowcolor{lightgray}{\bf Segnale}&{\bf Valore}&{\bf Rappresentazione }\\\cline{1-3}
A&30.\+769230769230795&Q\textsubscript{5,18} \\\cline{1-3}
B&0.\+3&Q\textsubscript{-\/1,24} \\\cline{1-3}
C&0.\+0049&Q\textsubscript{-\/7,30} \\\cline{1-3}
prim&25&Q\textsubscript{5,0} \\\cline{1-3}
\end{TabularC}
La rappresentazione ottimale per i segnali Sum1 e Sum2 è stata scelta in base a valori trovati empiricamente con 10\+M test preliminari. \begin{TabularC}{3}
\hline
\rowcolor{lightgray}{\bf Segnale}&{\bf Valore}&{\bf Rappresentazione }\\\cline{1-3}
Sum1&\mbox{[}-\/3; 189\mbox{]}&Q\textsubscript{5,18} \\\cline{1-3}
Sum2&\mbox{[}-\/0.\+09; 3\mbox{]}&Q\textsubscript{2,21} \\\cline{1-3}
\end{TabularC}
Come per i segnali precedenti, la rappresentazione per m e per q è stata scelta in base a valori trovati empiricamente con 10\+M test preliminari. \begin{TabularC}{3}
\hline
\rowcolor{lightgray}{\bf Segnale}&{\bf Valore}&{\bf Rappresentazione }\\\cline{1-3}
m&\mbox{[}-\/27; 606\mbox{]}&Q\textsubscript{10,13} \\\cline{1-3}
q&\mbox{[}-\/2.\+62; 2.\+59\mbox{]}&Q\textsubscript{2,21} \\\cline{1-3}
\end{TabularC}
